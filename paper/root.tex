
\documentclass[letterpaper, 10 pt, journal, twoside]{ieeetran}
% \documentclass[lettersize,journal]{IEEEtran}
% \documentclass[lettersize,journal]{IEEEtran}
\newcommand{\T}{\mathsf{T}}
\IEEEoverridecommandlockouts                              % This command is only needed if 
                                                          % you want to use the \thanks command



\usepackage{amsmath,amsfonts}
\usepackage[ruled,vlined]{algorithm2e}
% \usepackage{algorithmic}
% \usepackage{algorithm}
\usepackage{array}
\usepackage[caption=false,font=normalsize,labelfont=sf,textfont=sf]{subfig}
\usepackage{textcomp}
\usepackage{stfloats}
\usepackage{url}
\usepackage{verbatim}
\usepackage{graphicx}
\usepackage{cite}
\hyphenation{op-tical net-works semi-conduc-tor IEEE-Xplore}
\usepackage[ruled,vlined]{algorithm2e}
% updated with editorial comments 8/9/2021
\usepackage{tikz}
\usetikzlibrary{shapes.geometric, arrows, positioning, calc, shapes.symbols}
\usepackage{booktabs}
%%%%%% My Packages %%%%%%
% \usepackage{subcaption}
% \usepackage{stfloats}
\usepackage{multirow}
\usepackage{lscape} % for large table
\usepackage{tabularx} % for large table
\usepackage{comment}
\usepackage{graphicx}
% \usepackage{subfigure,wrapfig}
%\usepackage{subfigmat}
\usepackage{color}
\usepackage{algorithm}
\usepackage[noend]{algpseudocode} % algorithmic, \State{}, etc
\usepackage{amssymb,amsmath}
\usepackage{amsthm} % theorem, lemma, etc
\usepackage{xspace}
\usepackage{hyperref}

\newtheorem{assumption}{Assumption}
\newtheorem{theorem}{Theorem}
\newtheorem{lemma}{Lemma}
\newtheorem{definition}{Definition}
\newtheorem{corollary}{Corollary}
\newtheorem{problem}{Problem}
% \newtheorem{proof}{Proof}
\newtheorem{remark}{Remark}
%\usepackage{enumitem}
% \usepackage{natbib}
%%%%%% End of My Packages %%%%%%

\newcommand{\blue}{\color{blue}}
\newcommand{\green}{\color{green}}
\newcommand{\citep}{\cite}

\newcommand\abbrAlg{XXX\xspace}
\newcommand\abbrAlgg{YYY\xspace}


% \ifthenelse{\boolean{shortver}}{% 
%     \thispagestyle{empty}
%     \pagestyle{empty}
% }{
%     \thispagestyle{plain}
%     \pagestyle{plain}
%     \pagenumbering{arabic}
% }

 
\begin{document}


\title{Fast Differential Dynamic Programming for Time-Optimal Trajectory Planning 
}

% Make room for more info lines in the \author command 
\author{Miaomiao Dai$^{1}$, Zhongqiang Ren$^{1}$%
}
% Use only for final RAL version.


% \author{Zhongqiang Ren$^{1}$, Sivakumar Rathinam$^{2}$ and Howie Choset$^{1}$
% \thanks{Manuscript received July, 19, 2022; Revised March, 8, 2023; Accepted April, 4, 2023. This paper was recommended for publication by Editor Francois Chaumette upon evaluation of the Associate Editor and Reviewers' comments. This work was supported by National Science Foundation under Grant No. 2120219 and 2120529. \textit{(Corresponding author: Zhongqiang Ren.)}
% }
% \thanks{Zhongqiang Ren and Howie Choset are with Carnegie Mellon University, 5000 Forbes Ave., Pittsburgh, PA 15213, USA. (email: zhongqir@andrew.cmu.edu; choset@andrew.cmu.edu).}%
% \thanks{Sivakumar Rathinam is with Texas A\&M University, College Station, TX 77843-3123. (email: srathinam@tamu.edu).}
% \thanks{Color versions of one or more figures in this article are available at}
% \thanks{Digital Object Identifier (DOI): }
% }

%\IEEEpubid{0000--0000/00\$00.00~\copyright~2021 IEEE}
%% Remember, if you use this you must call \IEEEpubidadjcol in the second
%% column for its text to clear the IEEEpubid mark.


% Paper headers
\markboth{IEEE Robotics and Automation Letters. Preprint Version. October, 2025}
{Ren \MakeLowercase{\textit{et al.}}: CP-MILP Mixed Integer Linear Programming for Multi-Agent Motion Planning.} 
% Use only for final RAL version

\maketitle

	\begin{abstract}
		\input{abstract}
	\end{abstract}

 
\begin{IEEEkeywords}
Motion and path planning, multi-robot systems, path planning for multiple mobile robots or agents. 
\end{IEEEkeywords}
	
	\graphicspath{{./figures/}}
	
	\section{Introduction}\label{milp:sec:intro}
	Time-optimal trajectory planning—generating motions that complete a task in the minimum possible time—is a fundamental requirement for agile robotic systems. From autonomous drone racing to emergency collision avoidance in self-driving cars, the ability to jointly optimize the control sequence and the total maneuver duration $T$ is critical for pushing physical limits. 
While Differential Dynamic Programming (DDP) and its variant, the iterative Linear Quadratic Regulator (iLQR), have become standard tools for high-dimensional trajectory optimization, they typically assume a fixed planning horizon. Extending these methods to time-optimal control introduces a discrete-continuous optimization challenge: the solver must determine the optimal integer horizon $T^*$ alongside the continuous control inputs.

Existing approaches to this problem generally fall into two categories: continuous-time relaxations and discrete search. 
Relaxation methods treat the final time as a continuous decision variable, typically by scaling the system dynamics. However, this often introduces non-convexity into the optimization landscape, leading to poor convergence.
Discrete search methods, on the other hand, treat $T$ as an integer parameter. A naive "brute-force" strategy involves evaluating every candidate horizon $T \in [T_{min}, T_{max}]$. 
\textbf{A fundamental bottleneck in this approach lies in the structure of the standard Riccati recursion.} In the LQR backward pass, the Value Function $V_k$ is computed recursively starting from a terminal cost anchored at the final time step $T$ (i.e., $P_T = Q_T$). Consequently, changing the horizon from $T$ to $T+1$ shifts the boundary condition, invalidating the entire sequence of previously computed Cost-to-Go matrices (P). This structural dependency prevents the reuse of historical computations across different horizons, forcing the solver to restart the backward pass from scratch for each candidate $T$, resulting in a prohibitive $\mathcal{O}(N^2)$ complexity.

To mitigate this computational burden, recent works such as the "One-Pass" method [1] have attempted to estimate costs for neighboring horizons by reusing the value function from a single nominal backward pass. While efficient for Linear Time-Invariant (LTI) systems where the dynamics do not shift with time, this approach fails for general trajectory optimization. In nonlinear systems, the local linearization ($A_k, B_k$) is time-varying; thus, reusing a fixed value function for different horizons introduces severe approximation errors, often leading to suboptimal horizon selection.

In this work, we propose a method that enables \textbf{exact computational reuse} for time-varying systems and extends it to the iLQR framework. 
Our approach builds on the observation that the Riccati difference equation can be viewed as a Linear Fractional Transformation (LFT). By reformulating the backward pass in information form, we construct a "propagator" that allows us to compose the inverse value functions incrementally. This structure decouples the backward recursion from the specific terminal time $T$, allowing us to query the optimal cost for \textit{any} candidate horizon using the same pre-computed propagator sequence. 
Furthermore, to apply this logic to iLQR, we introduce an augmented state space formulation that absorbs the time-varying affine linearization terms. This unifies the treatment of linear and nonlinear problems, allowing the propagator to compute the \textit{exact} LQR cost for all horizons in a single $\mathcal{O}(N)$ pass.

The main contributions of this paper are:
\begin{enumerate}
    \item \textbf{Propagator-based Horizon Selection:} We develop an LFT-based solver that enables the reuse of backward pass computations, reducing the complexity of horizon selection from $\mathcal{O}(N^2n^3)$ to $\mathcal{O}(Nn^3)$.
    \item \textbf{Augmented State Formulation:} We propose a state augmentation technique that embeds affine linearization terms into a homogeneous coordinate system, extending the efficient propagator method to general nonlinear iLQR problems.
    \item \textbf{Performance and Robustness:} We validate our algorithm on four benchmark systems, including a 12-DOF Quadrotor. Experimental results show that our method achieves speedups of up to $43\times$ compared to brute-force search while guaranteeing global optimality with respect to the linearized model. 
\end{enumerate}
	
	\section{Related Work}\label{milp:sec:related}
	\input{related}
	
	\section{Problem Description}\label{milp:sec:problem}
	% \subsection{Notation}
Let $x_{k+1} = f(x_k, u_k)$ denote the discrete-time dynamics of a system where $x_k \in \mathbb{R}^n$ and $u_k \in \mathbb{R}^m$ are the state and control vectors at time step $k \in \{0, 1, \ldots, T-1\}$, respectively, with $T \in \mathbb{N}$ being the planning time horizon.
Let $U_T=\{u_0, \ldots, u_{T-1}\}$ denote the sequence of all controls.
Let $\ell: \mathbb{R}^n \times \mathbb{R}^m \rightarrow \mathbb{R}_+$ and $\phi: \mathbb{R}^n \rightarrow \mathbb{R}_+$ denote the stage and terminal cost functions.
Let a non-negative real number $w \geq 0$ denote a weight factor.
This paper considers the following discrete-time optimal control problem with variable horizon.
\begin{problem}[General Problem]\label{tooc:problem:general}
\begin{equation}\label{eq:general_problem}
\begin{aligned}
\min_{U_T, T} \quad & J = \phi(x_T) + \sum_{k=0}^{T-1} \ell(x_k, u_k) + wT \\
\text{s.t.} \quad & x_{k+1} = f(x_k, u_k), \quad k = 0, \ldots, T-1 \\
& x_0 = \bar{x}_0 \\
& T \in \{1, 2, \ldots, N\}
\end{aligned}
\end{equation}
\end{problem}
Here, the decision variables include both the control sequence $U_T=\{u_0, \ldots, u_{T-1}\}$ and the finish time $T$.
The term $wT$ penalizes the planning horizon, and thus encourages time-optimal behaviour of the system.
When $w=0$, Problem~\ref{tooc:problem:general} becomes the regular optimal control problem with a fixed planning horizon~\cite{}.
When there is no stage and terminal costs $\phi(x_T)=0,\ell(x_k,u_k)=0, k=0,1,\cdots,T-1$ and $w > 0$, Problem~\ref{tooc:problem:general} becomes the regular time-optimal control problem~\cite{}.

We will begin with a simplified linear quadratic variant of Problem~\ref{tooc:problem:general}.
Let $A_k \in \mathbb{R}^{n \times n}$ and $B_k \in \mathbb{R}^{n \times m}$ denote the system and the input matrices, which can be time-varying as indicated by the subscript $k$. 
Let $Q_k \succeq 0$ and $R_k \succ 0$ denote the positive semi-definite state and positive definite control cost matrices, respectively.
Consider the case where the system is subject to linear dynamics $x_{k+1}=A_k x_k + B_k u_k$, and the stage cost $\ell(x_k,u_k) = x_k^\top Q_k x_k + u_k^\top R_k u_k$ and terminal cost $x_T^\top Q_T x_T$ are both quadratic.
Problem~\ref{tooc:problem:general} becomes the following Time-Optimal variant of the Linear Quadratic Regulator (TO-LQR) problem.

\begin{problem}[TO-LQR Problem]\label{tooc:problem:tolqr}
\begin{equation}\label{eq:lqr_cost}
\begin{aligned}
\min_{U_{T}, T} \quad & J = \tfrac{1}{2} x_T^\top Q_T x_T + \sum_{k=0}^{T-1}\tfrac{1}{2}\big(x_k^\top Q_k x_k + u_k^\top R_k u_k\big) + wT \\
\text{s.t.} \quad & x_{k+1}=A_k x_k + B_k u_k, \quad k = 0, \ldots, T-1 \\
& x_0 = \bar{x}_0 \\
& T \in \{1, 2, \ldots, N\}
\end{aligned}
\end{equation}
\end{problem}

When the dynamics and cost terms are nonlinear but twice continuously differentiable, Problem \ref{tooc:problem:general} can be iteratively approximated using Taylor expansion and solved using techniques similar to iterative LQR (iLQR).
\begin{assumption}\label{tooc:assume:differentiable}
    The cost terms $\phi,\ell$ are twice differentiable everywhere and the dynamics $f$ can be linearized everywhere.
\end{assumption}
For the rest of the paper, all our discussion on Problem \ref{tooc:problem:general} relies on this Assumption~\ref{tooc:assume:differentiable}.

\begin{remark}
Both Problem~\ref{tooc:problem:general} and \ref{tooc:problem:tolqr} can be modified to include an additional (soft) goal constraint $x_T=x_g$ where $x_g$ is the desired goal state to be reached by the system when $t=T$.
Our approach can handle these variants with some modification.
As opposed to this soft constraint on the goal state, an alternative way to include goal constraint is using soft constraint as described by the terminal cost $\phi$, which is common in the optimal control literature.
We therefore formulate the Problem~\ref{tooc:problem:general} and \ref{tooc:problem:tolqr} without this hard goal constraint.
\end{remark}

% We refer to the variants with this goal constraints corresponding to Problem~\ref{tooc:problem:general} and \ref{tooc:problem:tolqr} as Problem~\ref{tooc:problem:general} and \ref{tooc:problem:tolqr}


% \[
%   x_{k+1}=f(x_k,u_k),\qquad x_0~\text{given},
% \]
% with the optimization problem
% \[
%   \min_{\{u_k\},\,T\in\{1,\ldots,N\}} \phi(x_T)+\sum_{k=0}^{T-1}\ell(x_k,u_k)+wT.
% \]


% the general problem \eqref{eq:general_problem} reduces to the time-optimal LQR problem:
% \[
%   x_{k+1}=A_k x_k + B_k u_k,\qquad x_0~\text{given},
% \]
% with cost function
% \begin{equation}\label{eq:lqr_cost}
%   J_T = \tfrac{1}{2} x_T^\top Q_T x_T + \sum_{k=0}^{T-1}\tfrac{1}{2}\big(x_k^\top Q_k x_k + u_k^\top R_k u_k\big) + wT,
% \end{equation}
% where $Q_k\succeq 0$, $R_k\succ 0$. The decision variables are the control sequence $U_{0:T-1} = \{u_0, \ldots, u_{T-1}\}$ and the stopping time $T$.

% creating a trade-off between trajectory quality and time efficiency.

% Let $w > 0$ denote the per-step time penalty.
% Finally, let $\ell: \mathbb{R}^n \times \mathbb{R}^m \rightarrow \mathbb{R}_+$ and $\phi: \mathbb{R}^n \rightarrow \mathbb{R}_+$ denote the stage and terminal cost functions.

% \subsection{General Time-Optimal Control Problem}
% Consider the general discrete-time optimal control problem with variable horizon:
% \begin{equation}\label{eq:general_problem}
% \begin{aligned}
% \min_{u_0, \ldots, u_{T-1}, T} \quad & J = \phi(x_T) + \sum_{k=0}^{T-1} \ell(x_k, u_k) + wT \\
% \text{s.t.} \quad & x_{k+1} = f(x_k, u_k), \quad k = 0, \ldots, T-1 \\
% & x_0 = \bar{x}_0 \\
% & T \in \{1, 2, \ldots, N\}
% \end{aligned}
% \end{equation}
% where the decision variables include both the control sequence $\{u_0, \ldots, u_{T-1}\}$ and the stopping time $T$. The term $wT$ penalizes the execution time, creating a trade-off between trajectory quality and time efficiency.

% \subsection{Time-Optimal LQR (LTV)}
% Under the assumptions that:
% \begin{itemize}
%     \item The dynamics are linear time-varying: $f(x_k, u_k) = A_k x_k + B_k u_k$
%     \item The stage cost is quadratic: $\ell(x_k, u_k) = \frac{1}{2}(x_k^\top Q_k x_k + u_k^\top R_k u_k)$
%     \item The terminal cost is quadratic: $\phi(x_T) = \frac{1}{2} x_T^\top Q_T x_T$
% \end{itemize}
% the general problem \eqref{eq:general_problem} reduces to the time-optimal LQR problem:
% \[
%   x_{k+1}=A_k x_k + B_k u_k,\qquad x_0~\text{given},
% \]
% with cost function
% \begin{equation}\label{eq:lqr_cost}
%   J_T = \tfrac{1}{2} x_T^\top Q_T x_T + \sum_{k=0}^{T-1}\tfrac{1}{2}\big(x_k^\top Q_k x_k + u_k^\top R_k u_k\big) + wT,
% \end{equation}
% where $Q_k\succeq 0$, $R_k\succ 0$. The decision variables are the control sequence $U_{0:T-1} = \{u_0, \ldots, u_{T-1}\}$ and the stopping time $T$.

% \subsection{Time-Optimal iLQR}
% When the dynamics and costs are nonlinear but twice continuously differentiable, the general problem \eqref{eq:general_problem} retains its original form:
% \[
%   x_{k+1}=f(x_k,u_k),\qquad x_0~\text{given},
% \]
% with the optimization problem
% \[
%   \min_{\{u_k\},\,T\in\{1,\ldots,N\}} \phi(x_T)+\sum_{k=0}^{T-1}\ell(x_k,u_k)+wT.
% \]


    \section{Preliminaries}\label{milp:sec:preli}
    
\subsection{Fixed Horizon LQR Problem}
The conventional discrete LQR problem (with fixed planning horizon) can be solved to optimality via dynamic programming.
Let $V_k(x_k)= \frac{1}{2}x_k^\top P_kx_k$ denote the value function that describes the cost-to-go from state $x_k$ at time step $k$, where $P_k$ is computed backwards from $k=T$ to $k=0$.
When $k=T$, the cost-to-go is same as the terminal cost with $V_T(x_T) = \frac{1}{2}x_T^\top Q_T x_T$ and $P_{T} = Q_T$.
For other $k= \{1,2,\cdots,T-1\}$, the optimal solution is characterized by the backwards Riccati equations:
\begin{equation}\label{eq:riccati}
\begin{aligned}
S_k &= R_k + B_k^\top P_{k+1} B_k, \\
K_k &= S_k^{-1} B_k^\top P_{k+1} A_k, \\
P_k &= Q_k + A_k^\top P_{k+1} A_k - (A_k^\top P_{k+1} B_k) K_k.
\end{aligned}
\end{equation}
With a backward pass from $k=T$ to $k=1$, all $K_k,P_k$ matrices can be computed and the optimal control for each step can be obtained by $u_k^* = -K_k x_k$. 

\begin{figure}[tb]
    \centering
    \includegraphics[width=\linewidth]{source/figures/ti_reuse.png}
    \caption{\textbf{Time-invariant reuse.} The Riccati map $g$ is identical for all $k$, so one backward pass yields $\{P_k\}$ and we can read off $J_t$ by shifting the terminal index. This explains why horizon scanning is cheap when $g$ is time-invariant.}
    \label{fig:ti_reuse}
\end{figure}

\subsection{Varying Horizon Time-Invariant LQR Problem}
As opposed to fixing the planning horizon $T$ in LQR, when $T$ is also to be optimized, the existing approach in the literature~\cite{2021_OptimalHorizonDDP_Stachowicz} only considers the time-invariant case, and it is challenging to generalize this approach to the more general time-varying case.

Specifically, let $g$ denote a mapping corresponding to the computation process $P_k = g(P_{k+1})$ which obtains $P_k$ from $P_{k+1}$.
When the system dynamics satisfy $A_k=A,\forall k$, $B_k=B,\forall k$ (i.e., $x_{k+1}=A x_k + B u_k, \quad k = 0, \ldots, T-1$) and the cost terms satisfy $Q_k=Q,\forall k$, $R_k=R,\forall k$, the matrices $A, B, Q, R$ in the Ricatti equations are all constant for all time steps $k$.
As a result, the mapping $g$ becomes
\begin{align}
P_k &= g(P_{k+1}) \label{eq:Pk_map} \\
    &= Q + A^\top P_{k+1}
       \bigl[I - B(R + B^\top P_{k+1} B)^{-1} B^\top P_{k+1}\bigr] A . \nonumber
\end{align}
which is invariant across different time steps.

% ---------

This enables computational reuse: a single Riccati backward pass yields $\{P_0, P_1, \ldots, P_N\}$, from which we can extract the cost for any horizon $t$ as:
\begin{align}
J_t \approx \tfrac{1}{2} x_0^\top P_{N-t} x_0 + tw.
\end{align}
The minimum among all these $J_t, t=0,1,2,\cdots,T-1$ provides the optimal solution to the problem and the corresponding $t$ is the optimal horizon.

This result is already shown in optimal-horizon control \cite{2021_OptimalHorizonDDP_Stachowicz}.
We illustrate this computational process in Fig.~\ref{fig:ti_reuse}.
The same map $g(\cdot)$ is reused at every step, so one backward pass produces all $\{P_k, k=0,1,2,\cdots,T\}$. 
The colors emphasize horizon shifting.
For example, the blue case seeks $J_N$, uses $P_N$ as the terminal matrix (i.e., time index is $N\!+\!1$) and computes the cost via $P_0$, while the green case seeks $J_{N-1}$, uses $P_{N-1}$ as the terminal matrix (i.e., time index is $N$) and computes the cost via $P_1$.
% In general, changing the horizon corresponds to reading $P_{N-t}$ in $J_t$.

However, for time-varying systems, the mapping $g$ becomes $g_k$ that is also time-varying, and one would have to compute $g_k$ for each time possible horizon $k$.
As a result, the $P$-matrices $P_k$ cannot be reused since for different horizon $k$, the set of matrices $P_t, t=0,1,2,\cdots,k$ also varies and cannot be reused.
A naive approach is to solve for each possible horizon $k = 1,2,\cdots,N$ with a Riccati recursion, which leads to $N$ Riccati recursions in total and is computationally inefficient.
% this reusability of computed $P_k$ is lost.


    
	\section{Optimal Horizon Time-Varying LQR}\label{milp:sec:method}
	% \subsection{Method 1: Propagator-based Time-Optimal LQR via Linear Fractional Transformations}

% We present a novel approach for time-optimal control of linear time-varying (LTV) systems that achieves $\mathcal{O}(Nn^3)$ complexity—matching a single Riccati backward pass—while evaluating costs for all possible arrival times.

% \subsubsection{Key Insight and Motivation}




\begin{figure*}[t]
    \centering
    \includegraphics[width=0.9\textwidth]{source/figures/propagator_flow.png}
    \caption{\textbf{Time-varying propagator.} When $g_k$ varies with $k$, we switch to inverse form where each stage is an LFT $\tilde g_k$. The composed map $\tilde g_{0:k}$ remains an LFT with prefix parameters $(E_{0:k},F_{0:k},G_{0:k})$. This enables cheap horizon queries by reusing the composed mapping $\tilde g_{0:k}$ instead of reusing $\tilde P_k$ values.}
    \label{fig:propagator_flow}
\end{figure*}

% \subsection{Our Approach for solving time optimal TVLQR problems}\label{sec:method1}
% To address the loss of reusability in time-varying systems, we shift from \emph{reusing values} to \emph{reusing mappings}.
% In the time-invariant case, different horizons reuse the same Riccati update and only change which $P_{N-t}$ is read in $J_t$ (Fig.~\ref{fig:reuse_and_prop}(a)).
% In the time-varying case, $g_k$ changes with $k$, so the \emph{values} $\{P_k\}$ are not reusable across horizons.

To address this challenge, our key idea (Fig.~\ref{fig:propagator_flow}) is to rewrite the map $g_k$ as a new linear fractional transformation (LFT) form $\tilde{g}_{0:k}$ (which is explained later), and some of the matrices that help compute $\tilde{g}_{0:k}$ can be reused.
As a result, these matrices only need to be computed once for all possible horizons $k=1,2,\cdots,N$, as opposed to be repetitively computed for each possible horizon, which thus saves computational effort.

\subsection{Linear Fractional Transformation Form}
We first define necessary notations, then write down the results in Theorem~\ref{}, and finally provide the proof.

Let $\tilde P_k := P_k^{-1}$ denote the inverse matrix of $P_k$.
Let notation $\tilde g_{0:k}=\tilde g_0\circ\cdots\circ\tilde g_k$ denote a \textit{composed map} that composes the maps $g_0,g_1,\cdots,g_k$ sequentially.


\begin{theorem}[LFT form]
There exist matrices \((\overline{E}_{k},\overline{F}_{k},\overline{G}_{k})\), $k=0,1,2,\cdots,N$ such that
\begin{equation}\label{eq:prefix_LFT}
\tilde{g}_{0:k}(\tilde{P}) \;=\; E_{0:k} - F_{0:k}\,(\tilde{P}+G_{0:k})^{-1} F_{0:k}^\top,
\end{equation}
where, 
\begin{equation}\label{eq:prefix_recursion}
\begin{aligned}
W_k      &= (E_k + G_{0:k-1})^{-1},\\
E_{0:k}  &= E_{0:k-1} - F_{0:k-1} W_k F_{0:k-1}^\top,\\
F_{0:k}  &= F_{0:k-1} W_k F_k,\\
G_{0:k}  &= G_k - F_k^\top W_k F_k,\\
E_k &= Q_k^{-1},\qquad\\
F_k &= Q_k^{-1} A_k^\top,\qquad\\
G_k &= A_k Q_k^{-1} A_k^\top + B_k R_k^{-1}B_k^\top,
\end{aligned}
\end{equation}
with \(E_{0:0}=E_0,\;F_{0:0}=F_0,\;G_{0:0}=G_0\).
\end{theorem}

Note that $(A_k,B_k,Q_k,R_k)$ matrices are known as the input of the problem, and $(E_k,F_k,G_k,\overline{E}_k,\overline{F}_k,\overline{G}_k)$ are intermediate variables computed based on $(A_k,B_k,Q_k,R_k)$ and themselves recursively, and the composed map $\tilde{g}_k$ is computed based on $\overline{E}_k,\overline{F}_k,\overline{G}_k$ recursively.
We will explain this recursive computation later in Alg.~\ref{} with the help of Fig.~\ref{}.
We now prove the correctness of this theorem.

\begin{proof}
    .....
\end{proof}

\subsection{\abbrAlg Algorithm}

xxxxxxxxxxxx


------------------------------------------------


% make the per-step update \emph{composable} by working with the \textbf{inverse matrix}
% \[
% \tilde P_k := P_k^{-1},
% \]
% which we refer to as the \emph{inverse form}.
Under this change of variables, each stage becomes a linear fractional transformation (LFT)
$\tilde P_k=\tilde g_k(\tilde P_{k+1})$.
Although $\tilde P_k$ still depends on the terminal condition, the \emph{composed map}
$\tilde g_{0:k}=\tilde g_0\circ\cdots\circ\tilde g_k$ stays in the same LFT family and can be summarized by a prefix triple
$(E_{0:k},F_{0:k},G_{0:k})$ that depends only on stage data up to $k$.
Thus we build $(E_{0:k},F_{0:k},G_{0:k})$ once, and evaluate each candidate horizon by a single query
$\tilde P_0^{(t)}=\tilde g_{0:t-1}(\tilde P_t)$ (cf.~\eqref{eq:X0_from_prefix}), rather than re-running $t$ Riccati steps.



\subsubsection{Step 1: Transformation to inverse Form}

Rewrite \eqref{eq:riccati} as
\[
P_k
= Q_k + A_k^\top\!\Bigl[
P_{k+1}
- P_{k+1}B_k (R_k + B_k^\top P_{k+1}B_k)^{-1} B_k^\top P_{k+1}
\Bigr] A_k
\]
By the Woodbury identity
\[
(A+UCV)^{-1} = A^{-1} - A^{-1}U(C^{-1}+VA^{-1}U)^{-1}VA^{-1},
\]
we have
\[
\begin{aligned}
&P_{k+1} - P_{k+1}B_k (R_k + B_k^\top P_{k+1}B_k)^{-1} B_k^\top P_{k+1} \\
&= \bigl(P_{k+1}^{-1} + B_k R_k^{-1} B_k^\top\bigr)^{-1}.
\end{aligned}
\]
Hence
\begin{equation}\label{eq:P_via_X}
P_k
= Q_k + A_k^\top \bigl(P_{k+1}^{-1} + B_k R_k^{-1} B_k^\top\bigr)^{-1} A_k .
\end{equation}

Let \(\tilde{P}_k:=P_k^{-1}\) and \(U_k:=B_k R_k^{-1}B_k^\top\). Applying Woodbury again to \eqref{eq:P_via_X} yields the LFT form:
\begin{equation}\label{eq:LFT_step}
\begin{aligned}
\tilde{P}_k
&= \Bigl(Q_k + A_k^\top (\tilde{P}_{k+1}+U_k)^{-1} A_k \Bigr)^{-1}\\
&= Q_k^{-1}
   - Q_k^{-1} A_k^\top
     \Bigl( (\tilde{P}_{k+1}+U_k) + A_k Q_k^{-1} A_k^\top \Bigr)^{-1}
     A_k Q_k^{-1}.
\end{aligned}
\end{equation}

Define, for each \(k\),
\[
E_k := Q_k^{-1},\qquad
F_k := Q_k^{-1} A_k^\top,\qquad
G_k := A_k Q_k^{-1} A_k^\top + U_k .
\]
Then \eqref{eq:LFT_step} becomes
\begin{equation}\label{eq:LFT}
\tilde{P}_k \;=\; E_k \;-\; F_k\,(\tilde{P}_{k+1}+G_k)^{-1} F_k^\top
\;\equiv\; \tilde{g}_k(\tilde{P}_{k+1}),
\end{equation}
i.e., a \emph{linear fractional transformation} (LFT).

\subsubsection{Step 2: Closure Under Composition}

The LFT representation allows us to construct \textbf{propagators}—operators that 
efficiently propagate the value function inverse across multiple time steps. 
The key property is that these propagators can be composed without redundant computation.

Define the composed map
\(
\tilde{g}_{0:k}(\tilde{P}):= \tilde{g}_0\circ \tilde{g}_1\circ \cdots \circ \tilde{g}_k(\tilde{P}).
\)

\begin{theorem}
There exist matrices \((E_{0:k},F_{0:k},G_{0:k})\) such that
\begin{equation}\label{eq:prefix_LFT}
\tilde{g}_{0:k}(\tilde{P}) \;=\; E_{0:k} - F_{0:k}\,(\tilde{P}+G_{0:k})^{-1} F_{0:k}^\top .
\end{equation}
\end{theorem}

\begin{proof}
(Base case \(k=0\)) is immediate from \eqref{eq:LFT} with
\(E_{0:0}=E_0,\;F_{0:0}=F_0,\;G_{0:0}=G_0\).

(Inductive step.)
Assume \eqref{eq:prefix_LFT} holds for \(k-1\):
\[
\tilde{g}_{0:k-1}(\tilde{P})
= E_{0:k-1} - F_{0:k-1}\,(\tilde{P}+G_{0:k-1})^{-1}F_{0:k-1}^\top .
\]
Then
\[
\begin{aligned}
\tilde{g}_{0:k}(\tilde{P}) &= \tilde{g}_{0:k-1}\!\bigl(\tilde{g}_k(\tilde{P})\bigr) \\
&= E_{0:k-1} - F_{0:k-1} \Bigl(E_k - F_k(\tilde{P}+G_k)^{-1}F_k^\top \\
&\qquad + G_{0:k-1}\Bigr)^{-1}\! F_{0:k-1}^\top.
\end{aligned}
\]
Apply Woodbury to
\(
\bigl((E_k + G_{0:k-1}) - F_k(\tilde{P}+G_k)^{-1}F_k^\top\bigr)^{-1}.
\)
Let \(W_k:=(E_k + G_{0:k-1})^{-1}\). Then
\[
\begin{aligned}
&\bigl(W_k^{-1} - F_k(\tilde{P}+G_k)^{-1}F_k^\top\bigr)^{-1} \\
&= W_k + W_k F_k \bigl(\tilde{P} + G_k - F_k^\top W_k F_k\bigr)^{-1} F_k^\top W_k.
\end{aligned}
\]
Substituting and regrouping gives
\[
\begin{aligned}
\tilde{g}_{0:k}(\tilde{P}) &= \underbrace{E_{0:k-1} - F_{0:k-1}W_k F_{0:k-1}^\top}_{E_{0:k}} \\
&\quad - \underbrace{F_{0:k-1}W_k F_k}_{F_{0:k}} 
\Bigl(\tilde{P} + \underbrace{G_k - F_k^\top W_k F_k}_{G_{0:k}}\Bigr)^{-1} \\
&\qquad \times \underbrace{F_k^\top W_k F_{0:k-1}^\top}_{F_{0:k}^\top},
\end{aligned}
\]
which is of the desired form \eqref{eq:prefix_LFT}. 
\end{proof}

\paragraph{Explicit prefix recursion.}
Equivalently, the parameters obey (for \(k\ge 1\))
\begin{equation}\label{eq:prefix_recursion}
\begin{aligned}
W_k      &= (E_k + G_{0:k-1})^{-1},\\
E_{0:k}  &= E_{0:k-1} - F_{0:k-1} W_k F_{0:k-1}^\top,\\
F_{0:k}  &= F_{0:k-1} W_k F_k,\\
G_{0:k}  &= G_k - F_k^\top W_k F_k,
\end{aligned}
\end{equation}
with \(E_{0:0}=E_0,\;F_{0:0}=F_0,\;G_{0:0}=G_0\).




\subsubsection{Step 3: Efficient Query for All Arrival Times}

Given the prefix triple at \(t-1\), the initial inverse for any candidate
arrival time \(t\) and a terminal inverse \(\tilde{P}_t=(\alpha\tilde Q)^{-1}\) is
\begin{equation}\label{eq:X0_from_prefix}
\tilde{P}_0^{(t)} \;=\;
E_{0:t-1} - F_{0:t-1}\,(\tilde{P}_t + G_{0:t-1})^{-1} F_{0:t-1}^\top .
\end{equation}
Then \(P_0^{(t)}=(\tilde{P}_0^{(t)})^{-1}\) and
\begin{equation}\label{eq:Jt}
J_t \;=\; \tfrac12\,x_0^\top P_0^{(t)} x_0 \;+\; w\,t .
\end{equation}
Thus all \(\{J_t\}_{t=1}^N\) are obtained from a single forward prefix build
(using \eqref{eq:prefix_recursion}) plus \(N\) terminal updates
via \eqref{eq:X0_from_prefix}.

\subsubsection{Algorithm and Complexity Analysis}

\begin{algorithm}[t]
\SetAlgoLined
\DontPrintSemicolon
\caption{Propagator-based Time-Optimal LQR (LTV)}
\label{alg:propagator-ltv}

\KwIn{System matrices $\{A_k, B_k\}_{k=0}^{N-1}$, cost matrices $\{Q_k, R_k\}_{k=0}^{N-1}$, initial state $x_0$, terminal cost $\alpha\tilde{Q}$, time penalty $w$, horizon $N$}
\KwOut{Optimal costs $\{J_t\}_{t=1}^N$ for all possible arrival times}

\BlankLine
\tcp{\textbf{Phase 1: Build Prefix Propagators}}
\tcp{Step 1a: Compute inverse-form matrices}
\For{$k \gets 0$ \KwTo $N-1$}{
    Compute $Q_k^{-1}$ and $R_k^{-1}$\\
    $E_k \gets Q_k^{-1}$\\
    $F_k \gets Q_k^{-1} A_k^\top$\\
    $G_k \gets A_k Q_k^{-1} A_k^\top + B_k R_k^{-1} B_k^\top$\\
}

\BlankLine
\tcp{Step 1b: Accumulate prefix propagation}
Initialize: $\bar{E}_0 \gets E_0$, $\bar{F}_0 \gets F_0$, $\bar{G}_0 \gets G_0$\;

\For{$k \gets 1$ \KwTo $N-1$}{
    $W_k \gets (E_k + \bar{G}_{k-1})^{-1}$\;
    $\bar{E}_k \gets \bar{E}_{k-1} - \bar{F}_{k-1} W_k \bar{F}_{k-1}^\top$\;
    $\bar{F}_k \gets \bar{F}_{k-1} W_k F_k$\;
    $\bar{G}_k \gets G_k - F_k^\top W_k F_k$\;
}

\BlankLine
\tcp{\textbf{Phase 2: Compute Costs for All Time Steps}}
$\tilde{P}_T \gets (\alpha \tilde{Q})^{-1}$\tcp*{Terminal inverse matrix}

\For{$t \gets 1$ \KwTo $N$}{
    Retrieve prefix propagator: $(\bar{E}_{t-1}, \bar{F}_{t-1}, \bar{G}_{t-1})$\\
    $W_t \gets (\tilde{P}_T + \bar{G}_{t-1})^{-1}$\\
    $\tilde{P}_0 \gets \bar{E}_{t-1} - \bar{F}_{t-1} W_t \bar{F}_{t-1}^\top$\\
    $P_0 \gets \tilde{P}_0^{-1}$\\
    $J_t \gets \frac{1}{2} x_0^\top P_0 x_0 + t \cdot w$\;
}

\Return{$\{J_t\}_{t=1}^N$}\;
\end{algorithm}

\paragraph{Complexity and remarks.}
The propagator has the same order as a single LQR backward sweep,
\(\mathcal{O}(N n^3)\). Brute forcing all horizons by re-solving Riccati is
\(\mathcal{O}(N^2 n^3)\).

    
	\section{Optimal Horizon iLQR}\label{milp:sec:method2}
    
\subsection{Augmented State Space and Dynamics}

\subsection{\abbrAlgg Algorithm}

\subsection{Method 2: Augmented State Space for Time-Optimal iLQR}

Standard iterative Linear Quadratic Regulator (iLQR) is a trajectory optimization technique that solves nonlinear optimal control problems by iteratively approximating them as Linear Time-Varying (LTV) LQR sub-problems. In its classical form, iLQR operates on a fixed time horizon $N$. To extend this to time-optimal control where the horizon $T$ is a decision variable, we propose a strategy that embeds the efficient horizon selection developed in the previous section into the iLQR loop.

\paragraph*{Core Idea}
The central concept of our approach is to transform the local approximation generated by iLQR into a format compatible with the \textbf{Propagator-based Solver} (described in Section~\ref{sec:method1}). The process proceeds as follows:
\begin{enumerate}
    \item \textbf{Approximation:} At each iteration, we linearize the nonlinear dynamics and quadratize the cost function around the current nominal trajectory. This yields a local LTV optimal control problem with affine terms (due to linearization offsets).
    \item \textbf{Transformation:} We employ an \textit{Augmented State Space} formulation (introducing a homogeneous coordinate $z_k = [\delta x_k; 1]$) to absorb the linear and constant terms. This converts the affine LTV problem into a standard quadratic LQR form.
    \item \textbf{Horizon Selection:} With the problem now in standard form, we directly apply the Propagator method from Section~\ref{sec:method1} to efficiently evaluate the cost for all candidate horizons $t \in [T_{\min}, T_{\max}]$ and select the optimal $T^*$.
    \item \textbf{Update:} We perform the trajectory update using the feedback gains derived for this optimal horizon $T^*$.
\end{enumerate}

This formulation allows us to leverage the $O(N)$ complexity of the propagator method within the general nonlinear control framework.

\subsubsection{Linearization and Quadratization}

Given a nominal trajectory $(\bar{x}_k, \bar{u}_k)$, we define deviations $\delta x_k := x_k - \bar{x}_k$ and $\delta u_k := u_k - \bar{u}_k$. The linearized dynamics become:
\[
\delta x_{k+1} = A_k \delta x_k + B_k \delta u_k + a_k,
\]
where $A_k = \nabla_x f(\bar{x}_k, \bar{u}_k)$, $B_k = \nabla_u f(\bar{x}_k, \bar{u}_k)$, and $a_k = f(\bar{x}_k, \bar{u}_k) - \bar{x}_{k+1}$ captures the affine term.

The stage cost is expanded to second order:
\[
\begin{aligned}
\ell(&\bar{x}_k + \delta x_k, \bar{u}_k + \delta u_k) \approx \\
&\ell(\bar{x}_k, \bar{u}_k) + w + \ell_{x,k}^\top \delta x_k + \ell_{u,k}^\top \delta u_k \\
&+ \tfrac{1}{2} \delta x_k^\top \ell_{xx,k} \delta x_k + \delta x_k^\top \ell_{xu,k} \delta u_k \\
&+ \tfrac{1}{2} \delta u_k^\top \ell_{uu,k} \delta u_k,
\end{aligned}
\]
where subscripts denote partial derivatives evaluated at $(\bar{x}_k, \bar{u}_k)$, and the time penalty $w$ is included in the constant term.

\subsubsection{Pure Quadratization via Completing the Square}

The cross-term $\delta x_k^\top \ell_{xu,k} \delta u_k$ prevents direct application of standard LQR methods. To eliminate it and achieve pure quadratic form, we complete the square for control terms. All terms containing $\delta u_k$ are:
\[
\tfrac{1}{2} \delta u_k^\top \ell_{uu,k} \delta u_k + (\ell_{xu,k}^\top \delta x_k + \ell_{u,k})^\top \delta u_k.
\]

Define the shifted control variable:
\begin{equation}\label{eq:shifted_control}
v_k := \delta u_k + \ell_{uu,k}^{-1}(\ell_{ux,k} \delta x_k + \ell_{u,k}).
\end{equation}

After completing the square, the stage cost becomes:
\[
\begin{aligned}
\ell_k \approx& \; \tfrac{1}{2} \delta x_k^\top \underbrace{(\ell_{xx,k} - \ell_{xu,k} \ell_{uu,k}^{-1} \ell_{ux,k})}_{\text{modified } Q} \delta x_k \\
&+ \underbrace{(\ell_{x,k} - \ell_{xu,k} \ell_{uu,k}^{-1} \ell_{u,k})^\top}_{\text{modified linear term}} \delta x_k \\
&+ \tfrac{1}{2} v_k^\top \ell_{uu,k} v_k \\
&+ \underbrace{\left(\ell(\bar{x}_k, \bar{u}_k) + w - \tfrac{1}{2} \ell_{u,k}^\top \ell_{uu,k}^{-1} \ell_{u,k}\right)}_{\text{constant term with time penalty}}.
\end{aligned}
\]

\subsubsection{State Augmentation for Affine Terms}

To handle linear and constant terms systematically within the LQR framework, we augment the state with a unit element:
\[
z_k = \begin{bmatrix} \delta x_k \\ 1 \end{bmatrix}.
\]

This augmentation allows us to express all terms in pure quadratic form. Define the shorthand notation:
\[
\tilde{Q}_k = \ell_{xx,k} - \ell_{xu,k} \ell_{uu,k}^{-1} \ell_{ux,k}, \quad
\tilde{q}_k = \ell_{x,k} - \ell_{xu,k} \ell_{uu,k}^{-1} \ell_{u,k}.
\]
Then the augmented cost matrix becomes:
\begin{equation}\label{eq:aug_cost_matrices}
Q_k^{\text{aug}} = \begin{bmatrix}
\tilde{Q}_k & \tilde{q}_k \\
\tilde{q}_k^\top & 2\left(\ell(\bar{x}_k, \bar{u}_k) + w - \tfrac{1}{2} \ell_{u,k}^\top \ell_{uu,k}^{-1} \ell_{u,k}\right)
\end{bmatrix}.
\end{equation}
\[
R_k = \ell_{uu,k}.
\]

The stage cost in augmented form achieves the desired pure quadratic structure:
\begin{equation}\label{eq:aug_stage_cost}
\boxed{\ell_k \approx \tfrac{1}{2} z_k^\top Q_k^{\text{aug}} z_k + \tfrac{1}{2} v_k^\top R_k v_k}
\end{equation}

Similarly, the terminal cost matrix:
\[
Q_T^{\text{aug}} = \begin{bmatrix}
\phi_{xx,T} & \phi_{x,T} \\
\phi_{x,T}^\top & 2\phi(\bar{x}_T)
\end{bmatrix}.
\]

\subsubsection{Augmented Dynamics}

The linearized dynamics must also be transformed to work with the shifted control $v_k$:
\[
\delta x_{k+1} = \underbrace{(A_k - B_k \ell_{uu,k}^{-1} \ell_{ux,k})}_{\hat{A}_k} \delta x_k + B_k v_k \underbrace{- B_k \ell_{uu,k}^{-1} \ell_{u,k}}_{\tilde{a}_k}.
\]

The augmented system matrices that preserve the affine structure:
\begin{equation}\label{eq:aug_dynamics}
A_k^{\text{aug}} = \begin{bmatrix}
A_k - B_k \ell_{uu,k}^{-1} \ell_{ux,k} & -B_k \ell_{uu,k}^{-1} \ell_{u,k} \\
0 & 1
\end{bmatrix}, \quad
B_k^{\text{aug}} = \begin{bmatrix}
B_k \\ 0
\end{bmatrix}.
\end{equation}

\subsubsection{Integration with Propagator Method}

The key innovation is that the augmented system $(A_k^{\text{aug}}, B_k^{\text{aug}}, Q_k^{\text{aug}}, R_k)$ is now in standard LQR form, enabling direct application of the propagator method from Method 1. We compute:
\[
\begin{aligned}
E_k &= (Q_k^{\text{aug}})^{-1}, \\
F_k &= E_k (A_k^{\text{aug}})^\top, \\
G_k &= A_k^{\text{aug}} E_k (A_k^{\text{aug}})^\top + B_k^{\text{aug}} R_k^{-1} (B_k^{\text{aug}})^\top.
\end{aligned}
\]

Using the prefix composition formula (Eq.~\ref{eq:prefix_recursion}), we efficiently obtain propagator parameters $(\bar{E}_{t-1}, \bar{F}_{t-1}, \bar{G}_{t-1})$ for all candidate horizons. The cost for horizon $t$ is:
\begin{equation}\label{eq:horizon_cost}
J_t = \tfrac{1}{2} z_0^\top (\tilde{P}_0^{(t)})^{-1} z_0,
\end{equation}
where $\tilde{P}_0^{(t)} = \bar{E}_{t-1} - \bar{F}_{t-1}(\tilde{P}_T + \bar{G}_{t-1})^{-1} \bar{F}_{t-1}^\top$ and $\tilde{P}_T = (Q_T^{\text{aug}} + \varepsilon I)^{-1}$ with small regularization $\varepsilon > 0$.

The optimal horizon is selected as:
\[
T^* = \arg\min_{t \in [T_{\min}, T_{\max}]} J_t.
\]

This selection step, which would normally require $\mathcal{O}(N^2n^3)$ operations if solved independently for each horizon, now requires only $\mathcal{O}(Nn^3)$ using the propagator method.

\subsubsection{Truncated Backward Pass and Control Recovery}

After selecting $T^*$ using the propagator method, we perform a standard iLQR backward pass, but crucially only on the truncated interval $[0, T^*-1]$. This truncation provides significant computational savings compared to always computing over the full horizon $N$.

The backward pass computes value functions and feedback gains. Starting from terminal conditions at $T^*$:
\[
V_{xx,T^*} = \phi_{xx,T^*}, \quad V_{x,T^*} = \phi_{x,T^*}, \quad V_{0,T^*} = \phi(\bar{x}_{T^*}),
\]
we recursively compute Q-function derivatives and feedback gains as detailed in Algorithm~\ref{alg:time-optimal-ilqr}.

The feedback gain from the augmented system has the form:
\[
v_k = -K_k^{(v)} z_k, \quad \text{where} \quad K_k^{(v)} = [K_{x,k}^{(v)} \; k_{\rho,k}^{(v)}].
\]

To recover the original control deviation, we invert the transformation in Eq.~\eqref{eq:shifted_control}:
\begin{equation}\label{eq:control_recovery}
\delta u_k^* = -(K_{x,k}^{(v)} + \ell_{uu,k}^{-1} \ell_{ux,k}) \delta x_k - (k_{\rho,k}^{(v)} + \ell_{uu,k}^{-1} \ell_{u,k}).
\end{equation}
\begin{algorithm}[t]
\SetAlgoLined
\DontPrintSemicolon
\caption{Time-Optimal iLQR with Propagator Selection}
\label{alg:time-optimal-ilqr}

\KwIn{Dynamics $f$, costs $\ell, \phi$, initial state $x_0$, initial controls $U$ (or zeros), horizon bounds $[T_{\min}, T_{\max}]$, time penalty $w$}
\KwOut{Optimal trajectory and controls}

\Repeat{convergence}{
    \tcp{Step 1: Linearization and augmentation}
    Rollout $X$ using $f$ and $U$\;
    \For{$k = 0$ \KwTo $N-1$}{
        Compute $A_k, B_k$ at $(\bar{x}_k, \bar{u}_k)$\\
        Compute derivatives: $\ell_{x,k}, \ell_{u,k}, \ell_{xx,k}, \ell_{ux,k}, \ell_{uu,k}$\\
        Build $A_k^{\text{aug}}, B_k^{\text{aug}}, Q_k^{\text{aug}}, R_k$ via Eqs.~\eqref{eq:aug_cost_matrices}-\eqref{eq:aug_dynamics}\;
    }
    Build $Q_T^{\text{aug}}$ from $\phi$ at $\bar{x}_T$\;
    
    \BlankLine
    \tcp{Step 2: Propagator-based horizon selection}
    $J \gets$ PropagatorCost$(A^{\text{aug}}, B^{\text{aug}}, Q^{\text{aug}}, R, z_0, Q_T^{\text{aug}}, N)$\;
    $T^* \gets \arg\min_{t \in [T_{\min}, T_{\max}]} J[t]$\;
    
    \BlankLine
    \tcp{Step 3: Backward pass on $[0, T^*-1]$}
    Initialize: 
    
    $V_{xx}[T^*] \gets \phi_{xx}$, $V_x[T^*] \gets \phi_x$, $V_0[T^*] \gets \phi(\bar{x}_{T^*})$\\
    \For{$k = T^*-1$ \KwDownTo $0$}{
        \tcp{Compute Q-function derivatives}
        $Q_x \gets \ell_{x,k} + A_k^\top V_{x,k+1}$\\
        $Q_u \gets \ell_{u,k} + B_k^\top V_{x,k+1}$\\
        $Q_{xx} \gets \ell_{xx,k} + A_k^\top V_{xx,k+1} A_k$\\
        $Q_{ux} \gets \ell_{ux,k} + B_k^\top V_{xx,k+1} A_k$\\
        $Q_{uu} \gets \ell_{uu,k} + B_k^\top V_{xx,k+1} B_k$ (+ LM reg.)\;
        
        \tcp{Compute gains}
        $\kappa_k \gets -Q_{uu}^{-1} Q_u$, $K_k \gets -Q_{uu}^{-1} Q_{ux}$\\
        
        \tcp{Update value function}
        $V_{xx,k} \gets Q_{xx} - Q_{ux}^\top Q_{uu}^{-1} Q_{ux}$\\
        $V_{x,k} \gets Q_x - Q_{ux}^\top Q_{uu}^{-1} Q_u$\\
        $V_{0,k} \gets V_{0,k+1} + (\ell(\bar{x}_k, \bar{u}_k) + w) - \tfrac{1}{2} Q_u^\top Q_{uu}^{-1} Q_u$\\
    }
    
    \BlankLine
    \tcp{Step 4: Forward rollout with line search}
    \For{$\alpha \in \{1, 0.5, 0.25, 0.1\}$}{
        $x_{\text{new}}[0] \gets x_0$\\
        \For{$k = 0$ \KwTo $T^*-1$}{
            $\delta x \gets x_{\text{new}}[k] - \bar{x}_k$\\
            $\delta u \gets K_k \cdot \delta x + \alpha \cdot \kappa_k$\\
            $u_{\text{new}}[k] \gets \bar{u}_k + \delta u$\\
            $x_{\text{new}}[k+1] \gets f(x_{\text{new}}[k], u_{\text{new}}[k])$\;
        }
        \If{TrueCost$(x_{\text{new}}, u_{\text{new}}, T^*) <$ TrueCost$(\bar{x}, \bar{u}, T^*)$}{
            Accept trajectory; \textbf{break}\;
        }
    }
    \If{not accepted}{Increase LM regularization}
    \Else{Update nominal $(X, U)$}
}
\end{algorithm}
\subsubsection{Complete Algorithm}




Algorithm~\ref{alg:time-optimal-ilqr} presents the complete time-optimal iLQR procedure with propagator-based horizon selection. The algorithm alternates between:
1. Linearization and augmentation around the current nominal trajectory
2. Efficient horizon selection using the propagator method
3. Truncated backward pass on the optimal horizon
4. Forward rollout with line search for robust convergence

The integration of the augmented state space formulation with the propagator method is the key to achieving computational efficiency while handling the full nonlinear, time-optimal control problem.



\subsubsection{Complexity Analysis and Advantages}

The augmented propagator approach achieves significant computational advantages:

\begin{itemize}
    \item \textbf{Horizon selection}: $\mathcal{O}(Nn^3)$ for evaluating all candidate horizons using the propagator method, compared to $\mathcal{O}(N^2n^3)$ for brute-force evaluation
    \item \textbf{Per iteration}: Only one backward pass on the truncated interval $[0, T^*-1]$ instead of $N$ separate iLQR runs for different horizons
\end{itemize}

\textbf{Key innovation:} The augmented state space formulation transforms the nonlinear time-optimal control problem into a form amenable to efficient LFT-based computation. This unifies the treatment of linear and nonlinear cases—the nonlinear problem, after linearization and augmentation, can leverage the same efficient propagator machinery developed for the LTV case in Method 1. The result is a practical algorithm that achieves near-optimal time-optimal control for complex nonlinear systems with computational cost comparable to standard fixed-horizon iLQR.
    
	\section{Experimental Results}\label{milp:sec:result}
	
We validate our proposed Propagator-based iLQR on four benchmark systems: Double Integrator, Segway Balance, Cartpole Swing-Up, and a 12-DOF Quadrotor. 
We compare three horizon-selection strategies:
\begin{enumerate}
    \item \textbf{Ours (Propagator):} The proposed method using the Augmented State Propagator.
    \item \textbf{Baseline-1 (Bruteforce):} Evaluating all horizons via standard backward Riccati sweeps (ground truth).
    \item \textbf{Baseline-2 (OnePass):} The approximate strategy adapted from [1], using a single backward pass around a nominal horizon.
\end{enumerate}

% -----------------------------------------------------------------
% FIGURE 2: Experiment 1 (Aggregate Performance)
% -----------------------------------------------------------------
\begin{figure}[t]
    \centering
    \includegraphics[width=\linewidth]{source/figures/experiment_whole.png}
    \caption{\textbf{Experiment 1: Aggregate Performance.} 
    Comparison of (a) Normalized runtime (log scale) relative to the Brute-force baseline, and (b) Normalized optimal cost relative to the global optimum. Data represents statistics over 25 randomized trials.}
    \label{fig:results_main}
\end{figure}

\subsection{Experiment 1: Statistical Performance}
Fig. \ref{fig:results_main} summarizes the computational efficiency and solution quality across 25 randomized trials for each system.

\subsubsection{Computational Efficiency}
As shown in Fig. \ref{fig:results_main}(a), our Propagator method (Blue) achieves massive speedups compared to the Bruteforce baseline (Orange). 
Based on the aggregate data, our method achieves a median speedup of approximately $\mathbf{43.6\times}$ on the Cartpole Swing-Up task ($0.81$s vs $35.32$s) and $\mathbf{31.5\times}$ on the Segway Balance task ($0.11$s vs $3.47$s).

Notably, on the high-dimensional Quadrotor system, our method ($\approx 2.46$s) is not only over $3\times$ faster than Bruteforce ($\approx 8.21$s) but also faster than the OnePass heuristic ($\approx 2.88$s). This result highlights that the $\mathcal{O}(N)$ complexity of our LFT propagation scales more favorably with state dimension than the complex window-shifting logic required by OnePass.

\subsubsection{Optimality}
Fig. \ref{fig:results_main}(b) verifies the solution quality. Our method (Blue) perfectly aligns with the $y=1.0$ line across all benchmarks, empirically confirming that the Augmented State Propagator introduces \textbf{zero approximation error}.

In contrast, the OnePass method (Green) exhibits significant sub-optimality. Specifically, in the Cartpole Swing-Up task, it yields trajectories with a median cost increase of $\mathbf{7.2\%}$. This error stems from identifying an incorrect horizon ($T_{med}=140$ vs optimal $181$), indicating that reusing value functions from a single linearization point is unreliable for highly nonlinear maneuvers.

% -----------------------------------------------------------------
% FIGURE 3: Experiment 2 (Case Study / Landscape)
% -----------------------------------------------------------------
\begin{figure}[t]
    \centering
    \includegraphics[width=\linewidth]{source/figures/Quadrotor_Hover_Jt.png}
    \caption{\textbf{Experiment 2: Case Study on Quadrotor Hover.} 
    (Top) Comparison of cost landscapes ($J_t$) computed by different methods. (Bottom) Breakdown of total runtime into linearization, selection, backward, and forward phases.}
    \label{fig:quad_composite}
\end{figure}

\subsection{Experiment 2: Case Study on Cost Landscape}
To explain the performance gap observed in Experiment 1, we analyze the internal mechanics of a single Quadrotor Hover trial in Fig. \ref{fig:quad_composite}.

\subsubsection{Cost Landscape Analysis (Top Panel)}
The top panel compares the cost curves ($J_t$ vs. horizon $t$). 
The OnePass method (Purple) approximates the cost landscape by projecting the value function from a single nominal horizon. As visible in the plot, this approximation suffers from severe distortions (e.g., the artifact spike near $t=82$) when the system dynamics vary significantly over time. Consequently, it converges to a wrong local minimum at $T^*=84$, far from the true global optimum at $T^*=43$.

In contrast, our Propagator curve (Blue) overlaps perfectly with the Bruteforce markers (Yellow). This confirms that our augmented formulation correctly captures the exact time-varying LQR cost, allowing the solver to locate the true global optimum.

\subsubsection{Runtime Breakdown (Bottom Panel)}
The bottom panel decomposes the runtime to reveal the source of efficiency. 
The inefficiency of the Bruteforce method is visually evident in the massive Red block ("Select" phase), which represents the $\mathcal{O}(N^2)$ cost of repeated Riccati sweeps.
Our Propagator method effectively eliminates this bottleneck. By compressing the horizon evaluation into an $\mathcal{O}(N)$ LFT propagation, we reduce the selection time to negligible levels, achieving a total runtime of $2.46$s compared to $8.21$s for Bruteforce, while maintaining rigorous optimality.
    
	\section{Conclusion and Future Work}\label{milp:sec:conclude}
	\input{conclude}

\section*{Acknowledgements}

% 	\newpage
	\bibliographystyle{IEEEtran}
	\bibliography{references}

    % \clearpage
% 	\newpage


% 	\appendix
% 	\input{appendix}
 
%%%%%%%%%%%%%%%%%%%%%%%%%%%%%%%%%%%%%%%%%%%%%%%%%%%%%%%%%%%%%%%%%%%%%%%%%%%%%%%%
	
	
\end{document}
